\section{Introducción}

La Inmobiliaria www.barealestate.com.ar dispone de un sitio Web para la compra y
venta de inmuebles en la Ciudad de Buenos Aires. Para disminuir los costos de
publicación, necesita disponer de una pre-valorización de los inmuebles publicados a la
venta por los propietarios.

La empresa necesitaría de alguna forma estimar el rango del precio del metro cuadrado del
inmueble para decidir el nivel de especialista que destinará para realizar una
inspección física y así obtener una cotización real del precio de venta. Para la
empresa, el costo de la publicación depende mucho del nivel del especialista que tiene
que realizar la cotización, por lo tanto, desea bajarlo manteniendo acotado el riesgo de
una mala cotización.

Se usaron dos conjuntos de datos, uno para entrenamiento con 15.000 registros y otro con 4.000 registros para devolverlo clasificado, 
el conjunto de datos tiene 5 clases las cuales vienen dadas de la siguiente manera:

\begin{itemize}
 \item Clase 1: $m^2$ <= \$2.000
 \item Clase 2: \$2.000 < $m^2$ <= \$2.500
 \item Clase 3: \$2.500 < $m^2$ <= \$5.500
 \item Clase 4: \$5.500 < $m^2$ <= \$18.500
 \item Clase 5: \$18.500 < $m^2$
\end{itemize}

El objetivo de este trabajo práctico es desarrollar un modelo de predicción del rango
del precio del metro cuadrado de propiedades a la venta en la Ciudad de Buenos Aires, realizando
modelos de predicción teniendo en cuenta conceptos de árboles de decisión, minería de texto,
resampling y ensamble.