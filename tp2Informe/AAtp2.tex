\documentclass[11pt,conference]{IEEEtran}

\ifCLASSINFOpdf
   \usepackage[pdftex]{graphicx}
  % declare the path(s) where your graphic files are
   \graphicspath{{../pdf/}{../jpeg/}}
  % and their extensions so you won't have to specify these with
  % every instance of \includegraphics
   \DeclareGraphicsExtensions{.pdf,.jpeg,.png}
\else
  % or other class option (dvipsone, dvipdf, if not using dvips). graphicx
  % will default to the driver specified in the system graphics.cfg if no
  % driver is specified.
  \usepackage[dvips]{graphicx}
  % declare the path(s) where your graphic files are
  \graphicspath{{../eps/}}
  % and their extensions so you won't have to specify these with
  % every instance of \includegraphics
  \DeclareGraphicsExtensions{.eps}
\fi


\usepackage[utf8x]{inputenc}
\usepackage[T1]{fontenc}
\usepackage{lmodern}
\usepackage{booktabs}
\usepackage{multirow}
\usepackage{algorithmic}
\usepackage[Algoritmo]{algorithm}
\usepackage[spanish,USenglish]{babel}
\usepackage[colorlinks=true,linkcolor=black,urlcolor=black,citecolor=black, urlcolor=black, filecolor=black bookmarks=false]{hyperref}
\usepackage{subfigure}
\usepackage{array}
\usepackage{color}
\usepackage{graphicx}
\usepackage{epsfig}
\usepackage{multirow}
\usepackage{colortbl}
\usepackage[table]{xcolor}

\addto\captionsspanish{
 \def\tablename{Tabla}}


\begin{document}

\pagestyle{empty}  

\selectlanguage{spanish}

\title{Informe trabajo práctico número dos - Aprendizaje Automático}

\author{\IEEEauthorblockN{Omar Ernesto Cabrera Rosero}
\IEEEauthorblockA{Universidad de Buenos Aires\\
%Buenos Aires, Argentina\\
Email: omarcabrera@udenar.edu.co}
\and
\IEEEauthorblockN{Jimmy Mateo Guerrero Restrepo}
\IEEEauthorblockA{Universidad de Buenos Aires\\
%Buenos Aires, Argentina\\
Email: jimaguere@gmail.com}
}

\maketitle

%\selectlanguage{USenglish}
\begin{abstract}

En este trabajo pŕactico se desarrolla un modelo de predicción del rango del precio de $m^2$ de propiedades
a la venta en la Ciudad de Buenos Aires, de esta forma poder estimar el rango del precio del $m^2$ de un inmueble para
decidir el nivel de especialista que se destinará para realizar la inspección física y así obtener una cotización real
del precio en venta. Para la realización del modelo se tuvo en cuenta conceptos de árboles de decisión, resamplig,
ensamble y minería de texto.

\end{abstract}
 
\selectlanguage{spanish}


\begin{IEEEkeywords}
Árboles de decisión, clasificación, inmobiliaria, resampling, ensamble, minería de texto.
\end{IEEEkeywords}

\thispagestyle{empty} 

\IEEEpeerreviewmaketitle

\section{Introducción}

La Inmobiliaria www.barealestate.com.ar dispone de un sitio Web para la compra y
venta de inmuebles en la Ciudad de Buenos Aires. Para disminuir los costos de
publicación, necesita disponer de una pre-valorización de los inmuebles publicados a la
venta por los propietarios.

La empresa necesitaría de alguna forma estimar el rango del precio del metro cuadrado del
inmueble para decidir el nivel de especialista que destinará para realizar una
inspección física y así obtener una cotización real del precio de venta. Para la
empresa, el costo de la publicación depende mucho del nivel del especialista que tiene
que realizar la cotización, por lo tanto, desea bajarlo manteniendo acotado el riesgo de
una mala cotización.

Se usaron dos conjuntos de datos, uno para entrenamiento con 15.000 registros y otro con 4.000 registros para devolverlo clasificado, 
el conjunto de datos tiene 5 clases las cuales vienen dadas de la siguiente manera:

\begin{itemize}
 \item Clase 1: $m^2$ <= \$2.000
 \item Clase 2: \$2.000 < $m^2$ <= \$2.500
 \item Clase 3: \$2.500 < $m^2$ <= \$5.500
 \item Clase 4: \$5.500 < $m^2$ <= \$18.500
 \item Clase 5: \$18.500 < $m^2$
\end{itemize}

El objetivo de este trabajo práctico es desarrollar un modelo de predicción del rango
del precio del metro cuadrado de propiedades a la venta en la Ciudad de Buenos Aires, realizando
modelos de predicción teniendo en cuenta conceptos de árboles de decisión, minería de texto,
resampling y ensamble.
\input{metodologia.tex}
 
\section{Conclusiones}

Se construyón un modelo de clasificación para resolver el problema para predecir el precio del $m^2$  
a la venta en la Ciudad de Buenos Aires, de esta forma poder estimar el rango del precio del $m^2$ de un inmueble.

El árbol random forest, resultó ser el mejor clasificador de árboles debido a que se basa en el desarrollo de
muchos árboles de clasificación que dependen de un vector aleatorio probado independientemente y con la 
misma distribución para cada uno de estos.

El hacer resampling por filas sirve para generar varios modelos, y con ellos realizar votación, esto ayuda
a que al aplicar estos modelos, se pueda hacer una votación y se pueda quedar con el que valor con más frecuencia.

Cuando se aplica árboles de decisión algunos datos no se van a clasificar, por esta razón fue necesario utilizar
un clasificador probabilístico para los datos que no fueron clasificados.




%\appendices
%\section{Repositorio}
%El código fuente y conjunto de datos se encuentran en el repositorio de github.



% Can use something like this to put references on a page
% by themselves when using endfloat and the captionsoff option.
\ifCLASSOPTIONcaptionsoff
  \newpage
\fi


\bibliographystyle{IEEEtran}

\bibliography{IEEEabrv,bibliography}



\end{document}
